\documentclass[report.tex]{subfiles}
\begin{document}

\section{Introduction} % (fold)
\label{sec:introduction}
The application developed was done for the Android platform, with a target SDK
version of 25 and a minimum SDK version of 21.

For the purposes of this assignment, the following stages were attempted:
\begin{itemize}
    \item Stage 1: Basic activity goal setting and recording
    \item Stage 2: Basic activity history
    \item Stage 3: Test mode
    \item Stage 4: Enhanced activity goal setting and recording
    \item Stage 5: Enhanced activity history
    \item Stage 7: User preferences and settings
\end{itemize}

All of these stages were implemented to the fullest extent, with Stage 6 skipped
to provide more robustness to the already existing stages. The bonus stages were
not attempted for similar reasons.

\subsection{Design and Guidelines} % (fold)
\label{sub:design_and_guidelines}
The application was designed using Model-View-Presenter, which was inspired by
a GitHub repository provided by Google which demonstrates various architectures
(\url{https://github.com/googlesamples/Android-Architecture}). However,
improvements could be made since currently, the Presenters still require being
passed \emph{Context}s in the constructor since they are required for the
database objects.

Accessing data sources such as the SQLite databases and the user preferences
are done using a version of the Repository pattern. The Repository classes are
the only classes with knowledge of SQLite or \emph{SharedPreference}s, with
all other classes accessing the data through the repository's interface.

When developing on the SDK versions chosen, it was noted that the Java 8 time
APIs were not available on the platform, with the obsolete APIs still present.
Due to a personal preference over the new API, the library ThreeTenBP
(\url{https://github.com/ThreeTen/threetenbp}) was added for the project. This
is a backport based on JSR-10 (the new date and time library) meaning that I
was able to use manipulate dates and time in a more comfortable fashion.

The application sticks to Material Design as consistently as possible. The
colour scheme is consistent throughout the application, and appropriate elements
are used throughout. The application makes appropriate use out of 
\emph{RecyclerView}s, \emph{DialogPreference} as well.
% subsection design_and_guidelines (end)

\subsection{Feedback} % (fold)
\label{sub:feedback}
The feedback in Phase 2 of the assignment was incorporated at this stage. The
addition of goals is performed with a FloatingActionButton, while deleting goals
is not now. In addition, snackbars were used throughout the program indicating
messages to users. It is likely that there were other situations where they
would have been appropriate, but the essential ones were implemented.
% subsection feedback (end)
% section introduction (end)

\newpage
\end{document}